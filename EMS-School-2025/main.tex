\documentclass[11pt]{article}
\usepackage[margin=1in]{geometry}
\usepackage{lmodern}
\usepackage{amsmath, amssymb}
\usepackage{enumitem}
\usepackage{hyperref}
\hypersetup{
  colorlinks=true,
  linkcolor=blue,
  urlcolor=blue,
  pdftitle={Summer School Course Outline},
  pdfauthor={Your Name},
  pdfsubject={LVPP Method},
}

\title{\textbf{The Latent Variable Proximal Point Method:\\ How Information Geometry Can Provide New Solvers for Variational Inequalities, Nonlinear PDEs, and Beyond}}
\author{[Your Name] \\ \small [Your Affiliation] \\ \small \texttt{your.email@domain.edu}}
\date{Summer School Lecture Series — [Insert Dates Here]}

\begin{document}

\maketitle

\section*{Course Overview}
This three-part lecture series introduces the \textbf{Latent Variable Proximal Point (LVPP)} method, a novel approach leveraging information geometry to solve variational inequalities, nonlinear PDEs, and problems involving free discontinuities. Each 70-minute session builds progressively from foundational concepts to advanced applications and open directions.

\section*{Lecture Breakdown}

\subsection*{Lecture 1: Introduction, Preliminary Results, and the Obstacle Problem}
\begin{itemize}[leftmargin=1.5em]
  \item Motivation and context: Variational inequalities and PDEs
  \item A gentle introduction to information geometry
  \item Classical proximal point methods: strengths and limitations
  \item Setup and formulation of the obstacle problem
  \item Preliminary analytical tools: convexity, duality, monotonicity
\end{itemize}

\subsection*{Lecture 2: Derivation of LVPP, Application to the Obstacle Problem, Analysis and Experiments}
\begin{itemize}[leftmargin=1.5em]
  \item Derivation of the Latent Variable Proximal Point method
  \item Embedding the obstacle problem in the LVPP framework
  \item Convergence analysis and theoretical guarantees
  \item Numerical implementation and illustrative experiments
\end{itemize}

\subsection*{Lecture 3: LVPP for Free Discontinuity Problems, Nonlinear PDEs and Future Directions}
\begin{itemize}[leftmargin=1.5em]
  \item Extending LVPP to handle free discontinuities
  \item Applications to nonlinear PDEs and structured variational models
  \item Links to total variation and Mumford–Shah-type functionals
  \item Open problems and future research directions
\end{itemize}

\section*{Prerequisites}
Basic familiarity with:
\begin{itemize}[leftmargin=1.5em]
  \item Functional analysis and convex optimization
  \item Variational methods for PDEs
  \item A conceptual understanding of geometry or probability is helpful but not required
\end{itemize}

\section*{References and Resources}
Relevant papers, preprints, and code will be shared before each session. For background, participants may consult:
\begin{itemize}[leftmargin=1.5em]
  \item D. Kinderlehrer and G. Stampacchia, \textit{An Introduction to Variational Inequalities and Their Applications}
  \item Recent preprints and slides will be distributed via [insert platform or link]
\end{itemize}

\end{document}
