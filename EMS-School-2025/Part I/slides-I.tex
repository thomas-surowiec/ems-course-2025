%----------------------------------------------------------------------------------------
%	PACKAGES AND THEMES
%----------------------------------------------------------------------------------------
\PassOptionsToPackage{table}{xcolor}
\documentclass[aspectratio=169,xcolor=dvipsnames,11pt]{beamer}
\usetheme{SimplePlusAIC}
\usepackage{amsmath}
\usepackage{animate}
\usepackage{hyperref}
\usepackage{cleveref}
\usepackage{caption}
\usepackage{graphicx} % Allows including images
%\usepackage{subfig}
\usepackage{subcaption}
\usepackage{booktabs} % Allows the use of \toprule, \midrule and  \bottomrule in tables
\usepackage{svg} %allows using svg figures
\usepackage{tikz}
\usepackage{makecell}
\usepackage{multirow}
\usepackage{appendixnumberbeamer}
\usepackage{wrapfig}
\usepackage{verbatim}
\usepackage{tcolorbox}
%\usepackage[dvipsnames]{xcolor}

\usepackage{hhline}
\usepackage{relsize}
\usepackage{bm}
%Select the Epilogue font (requires luaLatex or XeLaTex compilers)
%\setsansfont{Epilogue}[
  %  Path=./epilogueFont/,
  %  Scale=0.9,
  %  Extension = .ttf,
   % UprightFont=*-Regular,
   % BoldFont=*-Bold,
   % ItalicFont=*-Italic,
    %BoldItalicFont=*-BoldItalic
    %]
    \usefonttheme[onlymath]{serif}
% \usepackage{ eulervm } % Euler VM as math serif font

\newcommand*{\defeq}{\stackrel{\text{def}}{=}}
\newcommand{\grad}{\nabla}
\newcommand{\lap}{\Delta}
\newcommand{\weaklyto}{\rightharpoonup}
\newcommand{\weakstar}{\stackrel{*}\rightharpoonup}
\newcommand{\cts}{\hookrightarrow}
\newcommand{\ctsDense}{\xhookrightarrow{d}}
\newcommand{\ctsCompact}{\xhookrightarrow{c}}
\newcommand{\E}{\mathbb{E}}
\newcommand{\pP}{\mathbb{P}}
\newcommand{\R}{\mathbb{R}}
\newcommand{\ER}{\overline{\mathbb{R}}}
\newcommand{\cR}{\mathcal{R}}
\newcommand{\cJ}{\mathcal{J}}
\newcommand{\cG}{\mathcal{G}}
\newcommand{\CVaR}{\textup{CVaR}}
\newcommand{\D}{\textup{ d}}
\newcommand{\dd}{\mathrm{d}}
\newcommand{\fa}{\text{for all }}
\DeclareMathOperator*{\essinf}{\vphantom{p}ess\,inf}
\DeclareMathOperator{\sigmoid}{expit} % a.k.a. logistic sigmoid

\usepackage[ruled,vlined,algo2e]{algorithm2e}
\crefname{algocf}{algorithm}{algorithms}
 \usepackage{caption}

\usepackage{tcolorbox}  % For fancy boxes

% Define a custom style for the box
\tcbuselibrary{skins, breakable}
% \newtcolorbox[auto counter, number within=section]{roundedshadowbox}[2][]{
%   enhanced,
%   colback=white, % Background color (kept white)
%   colframe=black, % Border color
%   boxrule=0.5pt, % Border thickness
%   arc=3mm, % Rounded corners
%   drop shadow southeast,
%   shadow xshift=0.5ex,
%   shadow yshift=-0.5ex,
%   colshadow=gray!60!white
%     #1 % Additional options (e.g., width override)
% }

% Define a custom gradient shadow box
% \tcolorboxenvironment{beamercolorbox}{
%   enhanced,
%   colback=white, % Background color (kept white)
%   colframe=black, % Border color
%   boxrule=0.5pt, % Border thickness
%   arc=3mm, % Rounded corners
%   drop shadow southeast,
%   shadow xshift=0.5ex,
%   shadow yshift=-0.5ex,
%   colshadow=gray!60!white
% }

%----------------------------------------------------------------------------------------
%	TITLE PAGE
%----------------------------------------------------------------------------------------

\title[\quad\quad\quad LVPP Course I]{The Latent Variable Proximal Point Method I: Introduction, Preliminary Results, and the Obstacle Problem
 } % The short title appears at the bottom of every slide, the full title is only on the title page
%\subtitle{Subtitle}

\author{\small{\bf Thomas M. Surowiec}}

\institute[T.M. Surowiec]{Department of Numerical Analysis and Scientific Computing \newline Simula Research Laboratory \newline Oslo, Norway}
% Your institution as it will appear on the bottom of every slide, maybe shorthand to save space


\date[EMS School]{ {\footnotesize 
K\'acov, Czechia, 15-20 June 2025}}
%----------------------------------------------------------------------------------------
%	PRESENTATION SLIDES
%----------------------------------------------------------------------------------------

\begin{document}

{
\setbeamertemplate{background canvas}{}
\frame{\titlepage}
}

\begin{frame}{Overview}
\tableofcontents
\end{frame}

\section{Motivation and context: Variational inequalities and PDEs}\label{sec:motivation}
\begin{frame}{Dirichlet's Principle}
    \begin{enumerate}
        \item History
            \begin{enumerate}
                \item Who invented it, when did it get its name, etc.
            \end{enumerate}
        \item Modern interpretation and a first variational inequality
            \begin{enumerate}
                \item Derivation of a variational inequality
                    \begin{enumerate}
                        \item The feasible set is nonempty, closed, and convex
                        \item Existence via direct method
                        \item The objective function is G\^{a}teaux differentiable
                        \item Derivation of the VI
                    \end{enumerate}
            \end{enumerate}
        \item Reformulation as a variational equation
            \begin{enumerate}
                \item Using a clever test function
                \item Use lifting discussion see Chouly 2023, Poisson.
            \end{enumerate}
    \end{enumerate}
\end{frame}


\begin{frame}{Dirichlet's Principle}
% \visible<2->{
  \begin{minipage}{0.48\textwidth}
  \begin{beamercolorbox}[rounded=true, shadow=true, wd=\textwidth]{block body}
  A
% We seek a minimizer $u$ of the energy functional:
% \begin{equation*}
% 	E(v)
% 	=
% 	\frac{1}{2}
% 	\int_\Omega |\nabla v|^2 \dd x
% 	-
% 	\int_\Omega v f \dd x
% 	\,,
% \end{equation*}
% over the \textbf{closed convex cone} defined by  
% \[
% K = \{ v \in H^1_0(\Omega) \mid v \geq 0 \text{~a.e.}\}.
% \] 
% (Displacements $v \ge 0$, the obstacle)
   \end{beamercolorbox}
    \end{minipage}
    % }

% \begin{frame}\frametitle{Dirichlet's Principle\footnote{\tiny  Due to William Thomson, 1st Lord Kelvin 1847 \& Gau\ss. Named after his teacher, Dirichlet, by Riemann. Hilbert used ``Dirichlet's principle'' 1900.}}

%  In contemporary language, this states that for any $f\in L^2(\Omega)$ and $g\in H^1(\Omega)$, the (weak) solution of Poisson's equation over a Lipschitz domain $\Omega \subset \mathbb{R}^n$,
% \begin{equation}
% \label{eq:PoissonEquation}
% 	-\Delta u = f
% 	\quad \text{in~} \Omega,
% 	\qquad
% 	u = g \quad \text{on~} \partial\Omega,
% \end{equation}
% is the \textbf{$H^1(\Omega)$-minimizer} of the Dirichlet energy,
% \begin{equation}
% \label{eq:DirichletEnergy}
% 	E(v)
% 	=
% 	\frac{1}{2}
% 	\int_\Omega |\nabla v|^2 \dd x
% 	-
% 	\int_\Omega v f \dd x
% 	\,,
% \end{equation}
% when confined to the \textbf{constraint set} $H^1_g(\Omega) = g + H^1_0(\Omega) = \{ v \in H^1(\Omega) \mid v = g \text{~on~} \partial \Omega\}$.
\end{frame}

\begin{frame}{Elliptic Variational Inequalities}
        \begin{enumerate}
        \item A slight twist on Dirichlet's principle
            \begin{enumerate}
                \item Conic constraint ($u \ge 0$)
                \item Follow Dirichlet above for derivation
            \end{enumerate}
        \item A second variational inequality: An obstacle problem
        \item A brief history of variational inequalities (of the first kind)
        \item The Lions-Stampacchia theorem
        \item Mignot's theorem
            \begin{enumerate}
                \item 
            \end{enumerate}
    \end{enumerate}
\end{frame}
\section{Examples}\label{sec:examples}
\begin{frame}{Examples of Variational Inequalities}
    \begin{enumerate}
        \item Signorini problem
        \item Stefan problem
        \item Black-Scholes
        \item Multiphase phase transition
        \item Elastoplastic torsion
        \item Ice sheet
        \item Diffusive wave equation
    \end{enumerate}
\end{frame}
\section{Alternative Formulations}\label{sec:complementarity}
\begin{frame}{A Complementarity Formulation}
        \begin{enumerate}
        \item Derivation of a complementarity problem
        \item Regularity of the multiplier
        \item Active, Inactive, and Biactive sets
        \item General form of multiplier?
    \end{enumerate}
\end{frame}
\section{Solution Algorithms}\label{sec:algorithms}
\begin{frame}{Algorithms}
    \begin{enumerate}
        \item Penalty
        \item Barrier
        \item Augmented Lagrangian
    \end{enumerate}
\end{frame}


% \begin{frame}{Frame Title}
    
% \end{frame}
% \begin{frame}{Frame Title}
    
% \end{frame}
% \begin{frame}{Frame Title}
    
% \end{frame}
% \begin{frame}{Frame Title}
    
% \end{frame}
% \section{}
% \begin{frame}{Frame Title}
    
% \end{frame}
% \begin{frame}{Frame Title}
    
% \end{frame}
% \begin{frame}{Frame Title}
    
% \end{frame}
% \begin{frame}{Frame Title}
    
% \end{frame}
% \begin{frame}{Frame Title}
    
% \end{frame}
% \section{}
% \begin{frame}{Frame Title}
    
% \end{frame}
% \begin{frame}{Frame Title}
    
% \end{frame}
% \begin{frame}{Frame Title}
    
% \end{frame}
% \begin{frame}{Frame Title}
    
% \end{frame}

\end{document}